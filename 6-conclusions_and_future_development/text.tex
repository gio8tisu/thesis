\documentclass[../main.tex]{subfiles}
\begin{document}

The use of the \gls{cyclegans} framework for digitally staining \gls{cm} slides
has been studied, as well as fully-convolutional models for speckle denoising.

This work supplements the one by \cite{Combalia2019} mainly with three
contributions:
\begin{enumerate*}[label=\arabic*)]
\item A way of measuring StainNN hallucinations is studied;
\item Different inference techniques for whole slides are developed and compared
and
\item Use of a Unet-like architecture for the StainNN which is able to more
accurately transform the structures from the source without creating new ones.
\end{enumerate*}

The denoising model seems to work from a \gls{ssim} point of view, but based on
visual inpection the results are not satisfactory. The cause may be the
training images, as they are from a different character ---\gls{fcm} instead
of \gls{rcm}--- and the noise model used to contaminate the images
may not be accurate.

In the case of the staining model, the UNet-like architecture is superior to the
residual one based on both the \gls{lbp} histogram distance and \gls{ssim}; but,
as Dr. Pérez mentions, it is essential to make sure the structures are always
preserved.

\subsection{Future development}

To tackle the main problem of the StainNN (nuclei elimination), a new model
should be trained with a loss that further penalizes this kind of behavior.
Maybe ``showing'' both the input and output to the discriminator could work,
or just by trying higher $\lambda_{cycle}$ or $\lambda_{identity}$ values
the model could learn to be more conservative.

The applied StainNN architectures have a large number of parameters, reducing
the number of parameters could make the transformation less prone to
hallucinations and would reduce inference time and memory.
Therefore, as a future development, finding a model that matches or improves
the performace of the presented models but using less parameters would
be beneficial.

The speckle noise model was selected based on similar works, a validation of
different noise models should be done to see which one applies better to this
problem. Also related to the speckle noise, a measure of how the denoising model
affects final transformation should be done.

Finally, the inference model could be further improved. As it can be seen in
figures \ref{fig:inference-comparison-scan0} and
\ref{fig:inference-comparison-scan3}, the edges are degraded; this is due to
the shape of the window, applying a special case window shape for the edges
and corners this would be solved.
\end{document}
