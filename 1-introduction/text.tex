\documentclass[../main.tex]{subfiles}
\begin{document}

\subsection{Context}
\label{sec:context}
\lipsum

\subsection{Project Background}
\label{sec:project-background}

In recent years, deep learning (DL) has significantly improved the
performance of computer vision systems in image classification, object
detection and segmentation tasks. Generative adversarial networks (GAN)
in particular have revolutionized generative tasks like image synthesis
and image-to-image translation. Image-to-image translation is the
problem where we have a source image and we want to generate an image
based on the source but with different characteristics.

The project is a continuation of a work \cite{Combalia2019} where the idea of
image-to-image translation is applied to medical imaging, specifically
to generation of histological images with the appearance of H\&E stained
slides using a confocal microscopy (CM) as the source.

Confocal microscopy has enabled rapid evaluation of tissue samples
directly in the surgery room significantly reducing the time of complex
surgical operations in skin cancer, but the output largely differs from
the standard H\&E slides that pathologists use to analyze tissue
samples.

Confocal microscopes can operate in two modes which highlight different
microscopic structures in the tissue.

In the previous work, a deep learning technique is proposed to combine
the two modes of the CM into a H\&E-like image to facilitate their
interpretation by untrained pathologists and surgeons.

The initial chosen architectures are: a CNN with a multiplicative
residual connection to reduce the noise and then a CycleGan to combine
the two CM modes into a digitally stained slide.

The current model has room for improvements: the despeckling neural
network sometimes produces undesired artifacts and de stain generative
network can create or eliminate structures on the output image. This
project will focus on understanding why this happens and will try to
solve this problems.

\subsection{Requirements and specifiations}
\label{sec:requirements-and-specifiations}
\lipsum

\subsection{Methods and procedures}
\label{sec:methods-and-procedures}
\lipsum

\subsection{Document Structure}
\lipsum

\printbibliography
\end{document}
