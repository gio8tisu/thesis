\documentclass[a4paper,12pt,titlepage,twoside]{article}
\usepackage[english]{babel}
\usepackage[utf8]{inputenc}
 
\title{Digitally Stained Confocal Microscopy Thorugh Deep Learning}
\author{Sergio Garcia Campderrich}
\date{October 2019}

\usepackage[utf8]{inputenc}

% Language and font encodings
\usepackage[enlish]{babel} %spanish, english, ...
\usepackage{lipsum}
\usepackage[utf8]{inputenc}
\usepackage[T1]{fontenc}
\usepackage{parskip}
\setlength{\parskip}{4mm}
\setlength{\footskip}{60pt}
\setlength{\headheight}{15pt}

%% Sets page size and margins
\usepackage[a4paper,top=3cm, bottom=3.2cm, inner=3cm, outer=2cm, footnotesep=1cm, heightrounded]{geometry}


%% Useful packages

% \usepackage{float}
% \usepackage{verbatim}
% \usepackage{amsmath}
% \usepackage{systeme}
\usepackage{graphicx} %Loading the package
\graphicspath{{images/}} %Setting the graphicspath
% \usepackage{multirow}
% \usepackage{caption}
% \usepackage{subcaption}
% \usepackage{wrapfig}
% \usepackage[colorinlistoftodos]{todonotes}
% \usepackage[colorlinks=true, allcolors=blue]{hyperref}
% \usepackage{titlesec}
% \usepackage{fancyhdr}
% \usepackage[firstpage]{draftwatermark}
% \usepackage{transparent}
% \usepackage{textcomp} %podem escriure ``o'' amb la comanda \textdegree també € amb \texteuro
% \usepackage[gen]{eurosym} %tb official en lloc de ``gen''
% \usepackage[section]{placeins} %Evita que les figures saltin de secció
% \usepackage{fancyref}
% \usepackage{array}
% \usepackage{longtable}
% \usepackage{mathtools}
% \usepackage{commath}
% \usepackage{scrextend}%Indentar un bloc sencer
% \usepackage{tgpagella}
% \usepackage{dtk-logos}

%%Comentaris
\begin{comment}
Aquí podem posar tots els comentaris que calgui sense anar posant %
REFERENCIAT
\label{marker}, \ref{marker} and \pageref{marker} \footnote{footnote text}

\begin{addmargin}[esq]{dta}
El text d'aquí incrementa els seus marges segons ``esq'' i ``dta''
\end{addmargin}
\end{comment}

%%Comença pàgina nova per cada Secció SI hi ha alguna cosa escrita a la anterior

\newcommand{\sectionbreak}{\cleardoublepage}


%%Caps de pagina i peus (E/O (even/odd), L/C/R (left/center/right) y H/F (header/footer))
\fancyhf{}
\fancyhead[ER]{Memòria}
\fancyhead[OL]{Podeu posar el títol aquí}
\fancyhf[ELH, ORH]{pàg. \thepage}
%\fancyfoot[C]{}
\fancyfoot[EL, OR]{
\includegraphics[scale=0.2]{logo_telecos_2018.png}}
\pagestyle{fancy}
\raggedbottom

\SetWatermarkText{\hspace{9mm}\transparent{0.15}\includegraphics[scale=2.5]{logo_telecos_2018.png}}
\SetWatermarkAngle{0}
\SetWatermarkLightness{1}



\begin{document}
\renewcommand{\refname}{Bibliografia}
\begin{titlepage}
    {\centering
    {\Huge Treball de Fi de Grau}\\
    \vspace{5mm}
    {\Large \textbf{Grau en Enginyeria de Tecnologies i Serveis de la Telecomunicació}}\\
    \vspace{20mm}
    \Huge \textbf{Títol}\\
    \vspace{10mm}
    \Huge\textbf{MEMÒRIA}\\
    \vspace{3mm}
    \Large\today\\  %Si e lloc del dia de la darrera edició es vol una data fixa, elimineu \today i poseu la data
    }
    \vspace{20mm}
    \hspace{2mm}
    \begin{tabular}{l@{ } l}
        \vspace{5mm}
        \Large \textbf{Author:} & \Large{Sergio García Campderrich} \\
        \vspace{5mm}
        \Large\textbf{Director:} & \Large{Verónica Vilaplana}\\

         \Large\textbf{Convocatòria: } & \Large{data}\\
    \end{tabular}\par
    \vspace{10mm}
    {\centering
    \includegraphics[scale=0.3]{logo_telecos_2018.png}\\
    {\Large Escola Tècnica Superior \\ d'Enginyeria de Telecomunicació de Barcelona}\\
    \vspace{3mm}
    \includegraphics[scale=0.4]{loco_upc.png}
    \par
    }
    \end{titlepage}
%\maketitle

\clearpage
\thispagestyle{empty}
\null\newpage
\pagenumbering{arabic}

\section*{Resum}
1 Pàgina.
\lipsum[3-8]
\clearpage\null\newpage

\tableofcontents
\listoffigures

